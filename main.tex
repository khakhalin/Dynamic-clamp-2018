\title{Busch, Khakhalin. Intrinsic temporal tuning}

%\documentclass[twocolumn]{article}
\documentclass{article}

\usepackage[utf8]{inputenc}
\usepackage[top=0.85in,left=1.5in,right=1.5in,footskip=0.75in]{geometry} % For one-column version

%\usepackage[top=0.85in,left=1in,right=1.0in,footskip=0.75in]{geometry} % For two-column version
% marginparwidth=2in

\usepackage{helvet} % Set font to Arial-like
\renewcommand{\familydefault}{\sfdefault} % Force this font

\usepackage[round, numbers, authoryear]{natbib} % Reference manager, round citations, unsorted (we keep refs.bib manually sorted)
% \usepackage[super]{natbib} % Nature-style citations
% \setcitestyle{citesep={,}} % For nature-style, comma instead of ;

%\usepackage[switch,pagewise]{lineno} % Numbered lines for two columns
\usepackage[pagewise]{lineno} % Numbered lines for one column
\usepackage{xcolor}
\renewcommand{\linenumberfont}{\normalfont\bfseries\small\color{lightgray}}

% \usepackage{hyperref} % good for urls, but it also URLizes references
\definecolor{linkcolor}{rgb}{0.2,0.6,0.7} % Neuron-style link color
\usepackage[colorlinks=true,citecolor=linkcolor,urlcolor=blue]{hyperref}%
%\usepackage{url}

% improves typesetting in LaTeX
\usepackage{microtype}
\DisableLigatures[f]{encoding = *, family = * }

% text layout
\raggedright
%\textwidth 6in 
%\textheight 9in
\setlength{\parindent}{0em}
\setlength{\parskip}{1em}

% adjust caption style
\usepackage[aboveskip=5mm,labelfont=bf,labelsep=period,singlelinecheck=off]{caption}

% this is required to include graphics
\usepackage{graphicx}

% Multicolumns
\setlength{\columnsep}{1cm}

% Titles
\usepackage{titlesec}
\titlespacing{\section}{0pc}{0.5pc}{0pc}

% Headers and footers
\usepackage{fancyhdr}
\pagestyle{fancy}
\rhead{Busch SE, Khakhalin AS. Temporal tuning in the tectum. Page \thepage}
\cfoot{} % To kill footer page numbers

\begin{document} % --------------------- Document start ----------------

%TC:ignore 
% The comment above is for texcount to ignore this text.
% It will ignore everything until the endignore pair, and so on.
% We need to get down to 4500 words in the main text.

%\linenumbers % Comment this to suppress line numbers

% title goes here:
%\twocolumn[
\begin{flushleft}
{\Large
\textbf\newline{Intrinsic temporal tuning of neurons in the optic tectum is shaped by multisensory experience}
}
\newline
% authors go here:
\\
Silas E. Busch\textsuperscript{1},
Arseny S. Khakhalin\textsuperscript{1,*}
\\
\bigskip
{1} Biology Program, Bard College, Annandale-on-Hudson, NY. 

* Correspondence: khakhalin@bard.edu

\section*{Abstract}
% 150 words (149 as or Aug 07)

Homeostatic intrinsic plasticity is often described as an adjustment of neuronal excitability to maintain stable spiking output. Here we report that intrinsic plasticity in the tectum of Xenopus tadpoles also supports temporal tuning, wherein neurons independently adjust spiking responses to fast and slow patterns of synaptic activation. Using the dynamic clamp technique, and five different types of visual, acoustic, and multisensory conditioning, we show that in tadpoles exposed to light flashes, tectal neurons became selective for fast synaptic inputs, while neurons exposed to looming and multisensory stimuli remained responsive to longer inputs. We also report a homeostatic co-tuning between synaptic and intrinsic temporal properties in tectal cells, as neurons that naturally received fast synaptic inputs tended to be most responsive to long-lasting synaptic conductances, and the other way around. These results expand our understanding of plasticity in the brain, and inform future work on the mechanisms of sensorimotor transformation.

% Acoustic stimuli had different effects when delivered alone, or when combined with visual stimuli.

%\textbf{Keywords}: homeostatic plasticity, intrinsic plasticity, temporal tuning, optic tectum, network development, dynamic clamp

\section*{Significance statement}
% 119 words (limit 120)
%TC:endignore
With the recent explosion of work in neural connectivity reconstruction and biologically inspired deep learning, most researchers concentrate on the topology of connections between neurons, rather than on differences in neuronal tuning. Here we show that in a sensory network in Xenopus tadpoles, different neurons are tuned, and respond stronger, to either short or long synaptic inputs. This tuning tended to be opposite to the actual dynamics of synaptic inputs each cell received, such that neurons that normally receive shorter inputs generated stronger spiking in response to longer testing currents, and the other way around. This observation shows that even in networks that don't generate oscillations, neurons reshape their temporal selectivity, to optimize their impact on distributed calculations. 
%TC:ignore 
\bigskip

\end{flushleft} % Only relevant for two-column documents, but doesn't hurt
%] % End of one-column region

\section*{Introduction}

% TODO LIST

% In Discussion, mention that studying post-hyperpolarization currents would be a great next step

% Intro: be more specific that we're not talking about STG or spinal (with refs). That in FF intrinsic is often underplayed, and that the same is tru for neuro-inspired machine learning. Check my own response to elife rejection, draw from there.

% Explain why homeostatic plasticity in a feature space makes a network more sensitive to unusual stimuli

It is often assumed, in fields as diverse as connectomics and machine learning, that the main difference between functional and dysfunctional neural networks lies in their connectivity \citep{takemura2014,hildebrand2017,bassett2017network,reimann2017}. Biological neurons, however, are tuned in ways that go well beyond adjusting one “strength” value per synapse: cells within the same network often demonstrate variability of activation thresholds \citep{kole2012}, production of bursts \citep{popovic2011}, inactivation by strong inputs \citep{bianchi2012}, and more. This diversity of tuning relies on coordinated changes of many parameters across a complex, multivariate landscape \citep{oleary2013}, as neuronal phenotypes are shaped by sensory experiences, and adjusted by modulatory inputs \citep{evans2015}. The dysregulation of intrinsic plasticity affects network dynamics \citep{tien2018}, and can lead to a loss of function \citep{marcelin2009}. And yet, with an obvious exception of oscillatory networks \citep{marder2011,picton2018control}, for many brain areas it is still unclear whether variation of intrinsic phenotypes serves as a defining aspect of network topology and architecture, or whether it is just a consequence of transfer function normalization \citep{titley2017}.

The optic tectum of the \textit{Xenopus} tadpole is an ideal model for exploring these questions: it is a highly malleable distributed network of about $10^4$ neurons \citep{pratt2013}, involved in stimulus discrimination and sensorimotor transformations \citep{dong2009,khakhalin2014}. In development, tectal neurons acquire diverse intrinsic and synaptic phenotypes, that are also shaped by sensory experiences \citep{xu2011,ciarleglio2015}. Circuits in the tectum can learn and reproduce the temporal dynamics of inputs to which they were exposed \citep{pratt2008}: a property that could in principle be achieved through synaptic changes alone \citep{lukovsevivcius2009}, but which is more likely to involve intrinsic temporal tuning \citep{narayanan2008,beatty2014}. Finally, tectal neurons exhibit strong Na channel inactivation, which seems to play a role in collision detection \citep{jang2016}, and is one of the targets for intrinsic plasticity \citep{bianchi2012}.

In this study, we asked two specific questions about the properties of intrinsic plasticity in tectal networks. First, we asked whether intrinsic plasticity is limited to changes in excitability, or whether it is more nuanced and can differentially adjust neuronal responsiveness to inputs with different dynamics. Second, we checked whether changes in intrinsic properties in the tectum are a homeostatic response to each cell's history of synaptic activation, or if they are independent of synaptic properties. In our previous large-scale census of tectal cells \citep{ciarleglio2015}, we observed no interaction between intrinsic and synaptic phenotypes, and also, despite an extensive search, we did not detect signs of temporal tuning (ibid,  figures 2, 4). However, we propose that the standard current-clamp protocols used in previous studies \citep{pratt2007,hamodi2014} were not adequate to detect important changes in the function of voltage-gated channels (see "Discussion"). 

Here we show that the intrinsic plasticity of tectal neurons goes beyond changes in average spikiness, and supports temporal selectivity that can be reshaped by sensory experience. Furthermore, we show that the tuning of intrinsic properties is coordinated with the duration of synaptic inputs received by each cell. These results rely on two methodological innovations. First, instead of using current injections, we employed the dynamic clamp technique, which allowed a more realistic simulation of synaptic conductances \citep{prinz2004}. Unlike for more common voltage and current clamp techniques, in dynamic clamp, the electric current injected into the cell is dynamically adjusted, based on a predefined formula that depends on cell membrane potential and time. And second, instead of relying on one type of sensory stimulation, we used five different stimulation protocols, and compared their effects on intrinsic tuning. The use of different sensory modalities also gave us insight into an unrelated, but equally intriguing question of multisensory integration in the brain \citep{deeg2009,felch2016,truszkowski2017}, as for the first time we were able to look at the traces left by multisensory stimuli in the tectum of freely behaving tadpoles.



\section*{Results}

All analysis scripts, and summary data for every cell, can be found at:  \url{https://github.com/khakhalin/Dynamic-clamp-2018}

\subsection*{Changes in excitability in response to sensory stimulation}

Our first question was whether our stimulation protocols caused changes in intrinsic excitability of tectal neurons. From previous studies, we knew that in tadpoles exposed to four hours of LED flashes tectal neurons became more excitable \citep{aizenman2003,ciarleglio2015}. However, the stimuli we used were weaker, and similar to those used in behavioral experiments \citep{khakhalin2014,james2015,truszkowski2017}. We presented a checkerboard pattern that inverted once a second either instantaneously (Figure 1C left, dubbed “Flash”), or with a slow transition over the course of a second (old black squares shrank to white, while new black squares grew from old white squares; Figure 1C right; dubbed “Looming”; see “Methods”).

\begin{figure*}
\includegraphics[width=\linewidth]{fig1.png}
\caption{
Overview of experimental design, and summary of dynamic clamp results. (\textbf{A}). Positions of tectal neurons that were recorded. (\textbf{B}). Sample data from a dynamic clamp experiment. Bottom row: the dynamics of conductances G(t) of four different durations simulated by the dynamic clamp system. Middle row: the currents I(t) dynamically injected into a cell based on conductances of 4 different durations and 3 different amplitudes. Top row: resulting voltage traces V(t) that were recorded and analyzed. (\textbf{C}). A schematic of visual conditioning, in “Flash” (left) and “Looming” (right) groups. (\textbf{D}). The number of spikes produced by all neurons in all experiments, split by input peak conductance, and plotted against conductance duration. Black lines show respective averages. (\textbf{E}). A summary of data from (d), presented as averages and 95\% confidence intervals.}
\end{figure*}

After conditioning, we excised the brain, obtained whole cell patch clamp recordings from neurons in the tectum (Figure 1A), and counted spikes produced in dynamic clamp, in response to simulated synaptic conductances of different durations and amplitudes (Figure 1B). We used conductances of 4 different durations (100, 200, 500, and 1000 ms), and 3 different amplitudes (peak conductances of 0.2, 0.5, and 1.0 nS), matching the range of synaptic currents observed in tectal circuits \textit{in vivo} (see "Methods"). Contrary to our expectations, in tadpoles that were exposed to instantaneous checkerboard inversions (flashes), tectal neurons on average became not more, but less spiky, and generated 0.4$\pm$0.4 spikes, across all types of dynamic clamp inputs, compared to 0.9$\pm$1.0 spikes in control (Figure 1D, left column; F(1,677)=30.4, p=5e-8, n cells=28, 29. Here and below, F-values are reported for a multivariate fixed effects analysis of variance with selected interactions, where cell id is used as a repeated measures factor; see "Methods" for a detailed description). Neurons exposed to looming transitions also spiked less than control neurons (0.6$\pm$0.4 spikes; F(1,629)=9.6; p=0.002; n=28, 25), but more than those exposed to instantaneous "flashes" (F(1,641)=3.7; p=4e-6; n=29, 25), which was likewise unexpected. In the tadpole tectum, looming stimuli are known to elicit stronger responses compared to instantaneous inversions \citep{khakhalin2014}, yet the suppression yielded by looming stimuli was weaker.

We then looked at how an increase in transmembrane conductance translated into increased spike output of neurons, to map their amplitude tuning, or amplitude transfer function. Compared to control, neurons from animals exposed to visual stimuli had a flatter amplitude tuning curve, and did not increase their spiking as fast in response to larger conductances (F(1,677)=15.9, p=8e-5; and F(1,629)=8.3, p=0.004 for flash- and looming transitions respectively). The difference in amplitude tuning between these two types of stimulation was not significant (F(1,641)=3.5; p=0.06).

These results show that prolonged stimulation had an effect on intrinsic excitability, but its direction was opposite to what was previously described \citep{aizenman2003,ciarleglio2015}, as neurons became less excitable. Moreover, while looming stimuli are known to be more salient, both behaviorally and physiologically \citep{khakhalin2014}, they had a weaker long-term effect on neuronal excitability in comparison to less salient flashes. We conjecture that the difference in the direction of change is due to our stimuli being weaker than those used in earlier studies (see Discussion), and we further explore the difference between flashes and looming stimuli below.

\subsection*{Changes in intrinsic temporal tuning}

We then examined whether different types of sensory activation would differentially reshape temporal intrinsic tuning in tectal neurons. As changes in intrinsic properties seemed homeostatic (increased activation led to reduced spiking), by the same logic, one could expect shorter stimuli (flashes) to selectively suppress responses to shorter synaptic inputs. Alternatively, shorter stimuli could reshape the network, making neurons better adjusted to working with short bursts of activation, as  previously described for synaptic processing \citep{aizenman2007} and recurrent activity in the tectum \citep{pratt2007,shen2011}.

Across input conductances of different lengths (100, 200, 500, and 1000 ms), neurons exposed to flashes and looming stimuli responded differently than in control (F(1, 677)=25.1; p=7e-7; and F(1,629)=12.0; p=6e-4 respectively). In control neurons, longer inputs typically evoked stronger spiking, whereas neurons from stimulated animals had flatter tuning curves, with a plateau, or even a decrease in spike number for longer conductance injections (Figure 1E). In essence, while control neurons “preferred” longer inputs, stimulated neurons developed a preference for shorter synaptic inputs, and this change was more pronounced in neurons exposed to flashes than in those exposed to looming stimuli (F(1,641)=7.5; p=0.006). This suggests that the change in overall intrinsic excitability, and the change in temporal tuning, follow two different kinds of logic. The overall excitability is homeostatic, as neurons became less excitable in response to stronger stimulation. The temporal retuning however can be better described as “adaptive”, as neurons exposed to shorter stimuli (flashes) became relatively \textit{more} responsive to shorter stimuli, and less responsive to longer stimuli, which is opposite of what one would expect for a fully homeostatic retuning. We chose to call this type of plasticity "adaptive", as presumably it means that after exposure to faster stimuli, neurons become more equipped to process faster patterns of activation.

\subsection*{Effects of acoustic and multisensory stimulation}

While the optic tectum (homologous to superior colliculus in mammals) is often described as a primarily visual area, it is also involved in heavy multisensory computations \citep{stein2014}. In tadpoles, it integrates visual information with inputs from mechanosensory, auditory, and lateral line modalities \citep{deeg2009,pratt2009,hiramoto2009,felch2016,truszkowski2017}, but the logic of this integration is still unclear. We wondered whether acoustic stimuli would reshape intrinsic properties of tectal neurons, and whether this retuning would be similar to that produced by visual stimuli. 

To test this question, we exposed tadpoles to four hours of behaviorally salient “click” sounds \citep{james2015,truszkowski2017}, provided at the same frequency (every second) as visual stimuli in the first set of experiments. We found (Fig 1D, E) that exposure to startle-inducing sounds (group “Sound”) did not lead to significant changes in either average spikiness (0.8$\pm$0.7 spikes; F(1,689)=1.7; p=0.2; n=28, 30), amplitude transfer function (F(1,689)=2.0; p=0.2), or temporal tuning curve (F(1,689)=2.0; p=0.2). This suggests that acoustic stimuli did not strongly activate tectal circuits during conditioning, despite being more behaviorally salient (for our stimuli, at the onset of stimulation, acoustic clicks evoked startle responses in about 50-80\% of cases, compared to 5-10\% for checkerboard inversions \citep{james2015,truszkowski2017}). As one possible explanation, auditory and mechanosensory inputs may have different cellular or subcellular targets in the tectum \citep{bollmann2009}, or they may differently recruit tectal inhibitory circuits \citep{liu2016,hamodi2016}.

We then combined visual and acoustic stimuli in two different ways and looked at the effects of multisensory stimulation on the intrinsic properties of tectal neurons. For some animals, we synchronized the instantaneous checkerboard inversions (flashes) with sound clicks (dubbed “Sync”), while for others we staggered visual and acoustic stimuli by half a period (500 ms; dubbed “Async”). We found (Fig 1D, E) that, after four hours of multisensory stimulation, tectal neurons were more excitable than after visual stimulation alone (0.6$\pm$0.4 spikes, F(1,689)=11.2, p=8e-4; and 0.7$\pm$0.6 spikes, F(1,665)=41.7, p=2e-10, for sync and async respectively). Compared to the “Flash” group, the tuning curves in multisensory groups were less flat, with a stronger effect in the Async group (for amplitude tuning: F(1,689)=1.3, p=0.3, and F(1,665)=2.4, p=0.02 in Sync and Async groups respectively; for temporal tuning: F(1,689)=9.3, p=0.002, and F(1,665)=22.8, p=2e-6 respectively). This suggests that on their own sound clicks had little effect on tectal excitability, but when added to visual flashes, sound clicks negated effects of retuning that visual stimulation would have had (Fig 2C).

\begin{figure*}[!t]
\includegraphics[width=\linewidth]{fig2.png}
\caption{
Quantification of changes in temporal tuning in response to sensory experience. (\textbf{A}). An illustration of how “Temporal tuning” and “Amplitude tuning” values were calculated. For the temporal tuning measure, the value of zero corresponds to linear dependency (blue line), positive values - to an accelerating, supralinear curve (red), and negative values - to a plateau-shaped curve (purple, yellow). For amplitude tuning, higher values correspond to faster increase in spiking with increased conductance. (\textbf{B}). Amplitude tuning of neurons across different experimental groups. (\textbf{C}). Temporal tuning and average spikiness of neurons in different experimental groups; in each plot all neurons across all groups are shown in gray, while neurons from one target group are shown in color; means are shown as black dots; ellipses represent 95\% normal confidence regions. Two outliers (top right corner) are brought within the axes limits. (\textbf{D}). Same data as in (C), shown as averages for each group, with 95\% confidence intervals. Black arrows show the effects of sound clicks, when they were added to control, and when they were added to “Flashes”, to form two types of multisensory stimuli.}
\end{figure*}

\subsection*{Changes in average neuronal tuning, and tuning variability}

To visualize and interpret differences in neuronal tuning, we quantified each of the three aspects of intrinsic tuning (average spikiness, amplitude tuning, and temporal tuning) with one value per neuron (see Methods). We used the mean number of spikes across all conditions as the measure of "spikiness"; the linear slope of the number of spikes as a function of conductance amplitude as the measure of "amplitude tuning", and the quadratic term of the curvilinear regression for the number of spikes as a function of input duration as the value of "temporal tuning" (Figure 2A). The numerical values of these "tuning coefficients" are not easily interpretable, but they capture the character of tuning curves for each neuron (Figure 2A). All three parameters differed across experimental groups: F(5,160)=3.1, p=0.01 for average spikiness (Figure 2C; see "Methods" for model description); F(5,160)=4.8, p=4e-4 for amplitude tuning (Figure 2B); and F(5,160)=3.6, p=4e-3 for temporal tuning (Figure 2C).

A visual comparison of neuronal tuning in different experimental groups (Fig 2C) shows that acoustic stimulation had opposite effects when provided on its own (without visual stimulation) than when added to visual flashes. Compared to control neurons, cells exposed to sound had slightly lower amplitude tuning coefficients (flatter curve, Cohen’s d=$-$0.29), and more negative (curving down, d=$-$0.27) temporal tuning coefficients (Hotelling t-squared test p=0.2). When sounds were added to flashes, however, both transfer functions became less flat (d=0.65 and 0.26 for "Async" compared to “Flash”, for amplitude and temporal tuning respectively), and more like curves for control neurons (Hotteling test p=0.04). Thus, acoustic stimulation tended to tune the network in the same direction as visual stimulation when delivered alone (Figure 2D), but negated the effect of visual stimulation when combined with it. This may imply that multisensory integration in the tectum is dominated by inhibition (see Discussion). Note also that looming stimuli seemed to have weaker effects on neuronal tuning compared to flashes, both in terms of amplitude (d=0.44) and temporal tuning (d=0.22).

Describing neuronal tuning with only few variables allowed us to compare cell-to-cell variability of tuning in different experimental groups. We found that this variability decreased as neurons were modulated away from the baseline (Bartlett test p=2e-9 for amplitude tuning, p=1e-9 for temporal tuning; Figure 2D). Groups that were significantly different in average values also had different variances (F-test with p$<$0.05), such as Control vs. Flash and Control vs. Looming for both amplitude and temporal tuning. This expands on a finding in our previous study \citep{ciarleglio2015} that prolonged patterned stimulation reduces diversity of tuning profiles in the network, reshaping them according to the spatiotemporal characteristics of the stimulus, with stronger stimuli having stronger effects on tuning diversity.

\subsection*{Changes in synaptic properties}

To see whether prolonged sensory stimulation affected synaptic inputs received by tectal neurons, we recorded evoked excitatory postsynaptic currents in response to optic chiasm stimulation. We found that the amplitude of the early, monosynaptic component of evoked responses (the average current between 5 and 15 ms after the shock; Figure 3B) differed across experimental groups (Figure 3A; F(5,161)=3.2, p=0.009; see "Methods" for a description of the linear model we used). Both Sync and Async multisensory groups had larger early synaptic currents than the Control group (Tukey p=0.03 and 0.04; Cohen d=0.46 and 0.74 on log-transformed data respectively). The amplitude of late synaptic currents–produced by recurrent network activation (15-145 ms after the stimulus)–did not differ across groups.

\begin{figure*}[!t]
\includegraphics[width=\linewidth]{fig3.png}
\caption{Changes in synaptic transmission, and co-tuning of synaptic and intrinsic neuronal properties. (\textbf{A}). Amplitudes of early monosynaptic inputs to tectal neurons in all experimental groups (given in log scale, with outliers brought within the axes limits). (\textbf{B}). A sample synaptic recording, showing all traces (green) and an average trace (blue). The black bars show the areas at which early monosynaptic and late polysynaptic currents were measured; the vertical position of each bar represents the respective average current. The second, longer bar does not completely fit within the figure at this scale. (\textbf{C}). Synaptic current duration (vertical axis) was mostly defined by the amplitude of early monosynaptic inputs (horizontal axis). (\textbf{D}). Synaptic current duration was different between experimental groups (see the text). (\textbf{E}). Across experiments, average temporal tuning in each group positively correlated with the average durations of synaptic currents they received. (\textbf{F}). Within experimental groups, temporal tuning of individual neurons negatively correlated with the duration of synaptic currents they received. Axes show within-group deviations of temporal tuning and synaptic current duration from respective averages for each group. (\textbf{G}). Raw tuning and synaptic duration values for each treatment group, with linear regression lines.}
\end{figure*}

Neurons with different contribution of early and late synaptic responses naturally had different synaptic current duration, which we quantified for every cell as a “center of mass” of the current within the first 700 ms after optic chiasm stimulation (see “Methods”). As expected, cells with strong monosynaptic inputs had shorter synaptic currents, while polysynaptic activity made synaptic currents longer (Figure 3C; p=2e-16, r=$-$0.78, n=168). We found that the synaptic current duration was different across treatment groups (Figure 3D; F(5,163)=6.3, p=2e-5). Cells in Flash, Sync, and Async groups all received shorter synaptic inputs than Control cells (Tukey p$<$0.05, mean duration of 267$\pm$36, 253$\pm$39, 268$\pm$24 and 304$\pm$40 ms respectively), indicating that prolonged sensory activation with short, frequent stimuli, reshaped synaptic transmission in the tectum, making it faster, because of selective potentiation of visual inputs from the eye.

\subsection*{Co-tuning of synaptic and intrinsic properties}

To see whether intrinsic and synaptic temporal properties of tectal cells coordinated with each other, we compared intrinsic temporal tuning of every neuron (that is, whether it preferred longer or shorter simulated synaptic inputs in dynamic clamp experiments) to the actual duration of synaptic inputs it received during in-vitro stimulation of the optic chiasm. We found that on average, cells exposed to stronger sensory stimuli preferred shorter synaptic inputs, and also received shorter synaptic inputs, leading to a positive correlation between average synaptic and intrinsic properties for each group (Figure 3E; r=0.89, p=0.02, n=5). This means that \textit{between} treatment groups, changes in synaptic and intrinsic properties were adaptively coordinated, and the stronger was the average change in synaptic transmission, the more cells reshaped their intrinsic properties to adjust to this change.

In contrast, \textit{within} experimental groups, cells that preferred shorter synaptic inputs tended to receive longer synaptic currents, and vice versa (F(1,145)=4.9, p=0.03). We calculated differences between the properties of each individual cell and the average for the experimental group it belonged to, and showed these "deviations from respective averages" on one plot (Figure 3F). Cells that had shorter synaptic currents, compared to other cells in their group, tended to be selective for longer synaptic currents (adjusted r=$-$0.18, p=0.03, n=151). This means that individual neurons tended to tune their intrinsic properties away from the typical statistics of their inputs, increasing their ability to respond to unusual patterns of synaptic activation. 

%%% --- This section below was commented out in prior submissions

Note that the correlation of intrinsic and synaptic properties had opposite signs between-groups (positive) and within-groups (negative), so if we lumped all cells from all groups together, we would have found no correlation between these two variables (r=$-$0.04, p=0.7, n=151). This is a textbook example of a so called "Simpsons paradox", wherein a pattern holds within subgroups, but disappears or is reversed on a full set because of pronounced differences between groups.

To better understand the interplay between the stimulation-induced retuning of the entire tectal network, and tuning of individual cells within each network, we looked at the interaction of intrinsic and synaptic temporal properties separately for each treatment group (Figure 3G). The co-tuning seemed present in Control, Looming, and Sound groups (r=$-$0.11, $-$0.36, and $-$0.47), but was absent in groups that experienced flashes, and so were subjected to stronger retuning (r=0.27, $-$0.01, and 0.01 for Flash, Sync, and Async groups respectively). If three slow, and three fast (Flash-like) groups were combined into two larger "supergroups", the Control-Looming-Sound neurons exhibited a homeostatic negative correlation between synaptic duration and intrinsic temporal tuning (r=$-$0.29, p=0.01, n=70), while the Flash-Sync-Async neurons lacked cells preferring slow stimuli, and so lacked the co-tuning (r=0.08, p=0.50, n=80). This suggests that when fast visual stimulation made the tectal network selective to short activation patterns, it happened not because all neurons uniformly changed their properties, but because a particular subset of neurons, namely those that receive strong direct inputs from the retina and weak recurrent inputs, become less excitable. Normally these neurons experience fast synaptic inputs, but are most responsive to slow conductances; in overstimulated networks however they became selective to fast conductances.

For amplitude tuning, the interaction between synaptic and intrinsic parameters of tectal cells was inconclusive. The amplitude of early synaptic responses and intrinsic amplitude tuning formally correlated on a full dataset (p=0.03, r=$-$0.17, n=151), but the correlation disappeared (p$>$0.05) when the highly non-normal amplitude data was log-transformed, or when 4 extreme values (out of 135 total) were removed. When analyzed separately, the between-groups and within-groups correlations were also insignificant. A similar analysis for amplitudes of late synaptic currents also did not yield reliable results.

\subsection*{The mechanisms behind temporal intrinsic plasticity}

Knowing that tectal neurons can tune to inputs of different temporal dynamics, we then tried to identify the cellular mechanisms underlying this tuning. For each cell, we used a sequence of voltage steps (Figure 4A) to activate Na and K conductances, and quantified ionic current amplitudes and activation potentials (Figure 4B) as it was done in earlier studies \citep{ciarleglio2015}. Together with cell membrane resistance (Rm) and capacitance (Cm) it gave us eight intrinsic parameters for every cell: peak amplitudes for sodium current, early (transient) potassium current, and late (stable) potassium current (INa, IKt, IKs respectively), and activation potentials for these three currents (VNa, VKt, and VKs). 

\begin{figure}[!t]
\centering
\includegraphics[width=3.2in]{fig4.png}
\caption{Electrophysiological properties of individual neurons, compared to their spiking in current and dynamic clamp experiments. (\textbf{A}). Sample curves for a voltage step experiment; black bars show the areas used to average Na (bottom) and transient K (top) currents (see Methods). (\textbf{B}). Processing of ionic currents data, with IV-curves translated into two parameters (threshold potential and peak current) for each ionic conductance. (\textbf{C}). Sample data from a current clamp experiment; spikes are marked with black dots. (\textbf{D}). Estimations of spikiness from current clamp experiments (horizontal axis) and dynamic clamp experiments (vertical axis) correlate. (\textbf{E}). The number of spikes registered in current clump mode: values predicted from a linear model, plotted against actually observed values. The model works reasonably well (61\% of variance explained). (\textbf{F}). Similar comparison for the dynamic clamp experiments: the model has very low predictive value (13\% of variance explained). Here and in E, both values are adjusted for position.}
\end{figure}

We ran a stepwise generalized linear model selection analysis (R package stepAIC, \citealt{venables2013}) to explain the intrinsic tuning of cells recorded in all experimental groups through these eight variables. We found that the average spikiness (after compensation for position within the tectum; see Methods) was best described by a combination of sodium peak current (INa) and membrane resistance (Rm) variables, but these variables explained only 8\% and 2\% of cell-to-cell variance respectively (F(1,130)=11.3, and F(1,130)=4.5; Figure 3E). Together, all eight cellular parameters described only 13\% of variance in average spikiness. The temporal tuning value was best explained by sodium current activation potential and membrane resistance (VNa: 7\%, F(1,147)=10.7; Rm: 2\%, F(1,147)=3.1), with all eight variables explaining only 11\% of total variance. For amplitude tuning, the proposed best model included peak sodium current (INa: 6\%, F(1,163)=10.0) and sodium activation potential (VNa: 2\%, F(1,1,163)=3.5), with all eight variables accounting for 10\% of variance. This very low total explained variance suggests that while ionic currents and their activation potentials clearly affected intrinsic tuning of tectal cells, most cell-to-cell variability in intrinsic phenotypes stemmed from some other properties that were different between cells. In agreement with this assessment, the effect of experimental group on either mean spiking, temporal, or amplitude tuning curves remained significant even after compensating for all 8 intrinsic properties (sequential sum of squares analysis of variance p=0.03, 0.001, and 0.003 respectively), suggesting that changes in tuning across experimental groups was mediated by other factors.

\subsection*{A comparison between dynamic clamp and current clamp experiments}

The inability to predict spiking of tectal neurons through their isolated electrophysiological properties was unexpected, and stood in a seeming contradiction with our previous study \citep{ciarleglio2015}. Fortunately, in the current study, we recorded spiking traces in response to “classic” current steps (Figure 4C), which allowed a direct comparison between the results of current clamp and dynamic clamp protocols. Across all cells, the maximal number of spikes observed during current step injections correlated with the average number of spikes in dynamic clamp experiments (Figure 4D; r=0.46, p=2e-9, n=152). In agreement with \citep{ciarleglio2015}, spiking in current clamp experiments correlated with peak sodium (INa: r=0.42, p=2e-8, n=152) and stable potassium currents (IKs: r=0.39, p=2e-7), as well as activation potential for sodium current (VNa: r=0.24, p=0.02). Overall, the 8 intrinsic variables described above (Rm, Cm, three peak currents, and three activation potentials) explained 61\% of cell-to-cell variability in the maximal number of spikes from current clamp experiments (Figure 4E), comparable to 49\% reported in our previous study \citep{ciarleglio2015}, and noticeably higher than 13\% for dynamic clamp experiments (Figure 4F).

We can therefore conclude that our set of eight cellular parameters can better predict spiking during current step injections (61\% of variance) than in dynamic clamp experiments (13\% of variance). This suggests the existence of internal properties that strongly affect spiking in dynamic clamp experiments, but are inaccessible through standard current and voltage step protocols (see Discussion). Our hypothesis is indirectly supported by two more observations: that of the eight cellular parameters only one was significantly (p$<$0.05) different across treatment groups (NaV: F(5,175)=3.7, p=0.003), and that the number of spikes detected in current clamp experiments did not differ across experimental groups (F(5,165)=0.8, p=0.6). 

\subsection*{Effects of position within the tectum}

In all analyses presented above, we adjusted cell properties for rostro-caudal and media-lateral position of each cell within the tectum, as in tadpoles both intrinsic \citep{hamodi2014} and synaptic properties \citep{wu1996,khakhalin2012} are known to differ between older (rostro-medial) and younger (caudal, lateral) parts of the developing tectum. In this study, most cell properties we measured correlated (p$<$0.05, after correction for treatment group differences) with either medial or rostral position within the tectum (medial: membrane capacitance r=$-$0.17, membrane resistance r=0.25, sodium current activation potential r=$-$0.30, stable potassium current r=$-$0.16, early synaptic amplitude r=$-$0.22, synaptic current duration r=0.29; rostral: peak sodium current r=0.18, late synaptic amplitude r=0.11; n between 168 and 183). Curiously, neither of the three measures of intrinsic tuning (average spikiness, temporal tuning, and amplitude tuning) correlated with position (p$>$0.1, n=168). This may suggest that while low-level properties of tectal cells depended on their age, their spiking phenotypes were largely age-independent. Thus, different cells seemed to achieve similar spiking behaviors through different combinations of underlying parameters, relying on the principle of “parameter degeneracy” \citep{prinz2004degeneracy,drion2015}.


\section*{Discussion}

In this study we show that different sensory stimuli retuned neurons in the optic tectum of \textit{Xenopus} tadpoles in different ways, inducing changes in both their temporal tuning, and amplitude transfer functions. This addresses our first question about the functional scope of intrinsic plasticity in the optic tectum, and shows that it goes well beyond simple adjustments of neuronal spikiness.

As our technical resources are rather limited, in this study we don't explicitly address the mechanisms of newly discovered intrinsic temporal tuning. We can however offer two working hypotheses that may explain these results. As tadpole tectal cells don't express "true" resonance currents, such as h-currents \citep{ciarleglio2015}, most temporal tuning effects we observed seem to be due to differences in ionic currents inactivation in different cells. One obvious way of tuning ionic channel inactivation would be for every cell to modulate sodium or transient potassium currents via channel phosphorylation, or by changing the expression of different channel variants, to shift their inactivation dynamics \citep{frank2003nachannels,goldwyn2018a_current}. To test this hypothesis, one would need to induce changes in temporal tuning, similarly to how we did it in this paper, but then pharmacologically isolate different ionic currents, and then directly measure their inactivation dynamics.

Another hypothetical mechanism that can underlie our current results is inspired by a yet unexplained finding from our earlier study \citep{ciarleglio2015}: namely, that one of the key electrophysiological parameters regulating excitability of tectal cells is cell membrane capacitance (Cm). Traditionally, cell capacitance is thought of as a relatively immutable parameter that describes cell morphology, and can even be used to estimate its size, yet in the tadpole tectum, we found it to drop both with age, and after strong sensory stimulation, even though cells don't seem to change their visual appearance (\citealt{ciarleglio2015}, Figure 7D). An intriguing possibility is that these changes in cell capacitance may be due to electiric uncoupling of three major compartments of \textit{Xenopus} tectal cells: their dendritic arbor, soma, and axon initial segment \citep{bollmann2009,jarvis2018morphology}. This differential uncoupling may be achieved through either minor changes in cell morphology \citep{leterrier2018axon}, or through target modulation of sodium and potassium channels at key points between the compartments, which would introduce shunting, and so strongly affect cell excitability \citep{grubb2010activity,kuba2010initial}. One way to test this hypothesis could be to perform immunostaining of cleared tectum preparation after sensory stimulation, and if any changes in the distribution of sodium and potassium channels is discovered, to validate the effect of these changes in a computational model.

In answer to our second question, about whether intrinsic and synaptic properties of tectal cells are in any way coordinated, here we show that intrinsic and synaptic temporal properties are co-tuned in the tectum, and moreover, that this co-tuning can be modified by sensory experience. In contrast with earlier studies that report increased excitability post-stimulation \citep{aizenman2003,dong2009,ciarleglio2015}, we found that sensory stimulation led to a suppression of spiking. The reason for that, most probably, is that the visual stimulation used in earlier studies was provided with a LED box, which was so bright and of so high contrast that it caused a suppression of retinal synaptic inputs via a polyamine block of AMPA receptors \citep{aizenman2003}. This suppression then triggered a “second-order” homeostatic compensation \citep{turrigiano2011,tien2018}, making neurons spikier. In our current experiments, however, synaptic inputs were not suppressed, and neuronal activation during sensory conditioning was stronger than in control, causing a decrease in intrinsic spikiness.

Our study described an important difference between the dynamic clamp results (that were affected by stimulation, but could not be explained through low-level intrinsic parameters) and the results obtained in "classical" slow clamp experiments (that were not affected by stimulation, yet better coordinated with intrinsic parameters). This different may be interpreted in two ways. One possible interpretation is to assume that dynamic clamp in the soma provided a bad approximation of peripheral synaptic inputs, as space clamp error is more pronounced for fast voltage fluctuations than for constant current injection \citep{spruston1993,prinz2004}. We however find this hypothesis unlikely, as dynamic clamp responses were consistently different in animals with different sensory history, which suggests that at the very least, we have captured some important aspects of intrinsic diversity, even if our estimations were biased. 

Alternatively, and in our opinion more likely, intrinsic excitability of tectal cells is affected by properties that are not easily accessible by standard slow voltage and current clamp protocols, such as axon initial segment relocation, or targeted modulation of axonal voltage-gated channels we described above \citep{grubb2010activity,kole2012}. In cells with excitable dendrites, channels of the axon initial segment may constitute only a small fraction of all voltage-gated channels, yet have a disproportionately large effect on the spiking output of the cell, and on its temporal tuning \citep{kole2007,hamada2016}. Furthermore, prolonged current injections in the soma are likely to quickly inactivate transient channels in the axon, obscuring any possible interplay between action potential width and Na channels recovery during burst firing \citep{popovic2011,kole2012}. Converestly, this effect would still affect spiking in more realistic, fast dynamic clamp experiments.

Our findings lead to several verifiable predictions. As rapid inactivation of spiking in tectal neurons plays a role in collision detection \citep{khakhalin2014,jang2016}, a change in temporal tuning should affect collision detection dynamics, which can be verified experimentally. More specifically, we predict that visual stimulation that retunes neurons to faster stimuli (Flash) would make tadpoles selectively less responsive to slow collisions, and increase the latency of collision avoidance, due to non-linear dynamics of retinal activation for realistic looming stimuli that is slow in the beginning, and fast towards the end \citep{jang2016}. These changes in intrinsic temporal tuning would also reshape the connectivity of tectal networks, as fast-inactivating cells would not support short recurrent loops within the network, thus promoting long-ranged polysynaptic connectivity \citep{fiete2010stdp,clopath2010stdp}. Finally, based on the multisensory phenomena reported in this paper, we predict that even though multisensory stimulation tends to increase tectal responses in vitro \citep{felch2016,truszkowski2017}, it would be likely to reduce peak activation in vivo. 

% An internal comment: technically, Dong 2009 tested collision avoidance after visual stimulation, and got a somewhat opposite result: better avoidance of large dots (roughly equivalent to slow collisions), and worse avoidance to small dots (roughtly equivalent to fast collisions). But then, he used LED that we now know (Ciarleglio 2015) to have an opposite effect on intrinsic excitability, so it's hard to argue that Dong 2009 was a veryfication of our prediction. It is way too confusing.

To sum up, we present a novel case of temporal selectivity in non-oscillatory neurons in a distributed sensory network, and demonstrate that intrinsic temporal tuning of neural cells correlates with their synaptic properties. It is particularly interesting that the temporal co-tuning we observed was homeostatic in nature, as cells tended to be selective for inputs of dynamics they did not usually experience. We hypothesize that this adjustment of temporal tuning is the reason why this tuning was so easily disrupted by strong visual stimulation, when for a few hours we drastically changed the statistics of inputs received by every cell. Tuning to "unusual stimuli" at the level of individual neurons fits into the narrative of information transfer maximization \citep{brenner2000} and network criticality \citep{rubinov2011}, wherein every element of a network tries to locally maximize its influence over the overall computation. This may have intriguing consequences for the function and development of sensory networks in the brain, which can be further probed by computational modeling \citep{khakhalin2014,jang2016}, and verified in experiments. We also suspect that any cell with a sufficiently large dendritic tree would be able to tune its temporal intrinsic selectivity the way we describe. We expect that in most cases these changes would not be noticeable in experiments with standard voltage- and current-clamp protocols, but can be probed with a dynamic clamp technique. It would therefore be very interesting to see whether our results will replicate in other sensory systems, such as the mammalian cortex.

% tectal oscillations were described both in post-metamorphic frogs \citep{baranauskas2012}, and in non-amphibian species \citep{sridharan2011,goddard2012,grossberg2016}.

\section*{Acknowledgements}

We would like to thank Carlos Aizenman (Brown University), Justin Hulbert (Bard College), and Kara Pratt (University of Wyoming) for their feedback on this paper drafts. Part of this work was supported by the Bard Summer Research Institute (BSRI) program.

% \section*{Declaration of Interests}

The authors have no conflicts of interest to disclose.

\section*{Author Contributions}

S.E.B.: Conception and design, acquisition of data, analysis and interpretation of data, drafting and revising the article. 

A.S.Kh.: Conception and design, analysis and interpretation of data, figure preparation, drafting and revising the article.

\section*{Materials and Methods}

% All data reported in this study, as well as the code used for its analysis, is available from the corresponding author (A.S.Kh.) upon request.

\subsection*{Housing and sensory conditioning}

All experimental protocols were in accordance with Bard College Institutional Animal Care and Use Committee (IACUC), and National Institutes of Health (NIH) guidelines. Animals were purchased from Nasco (Fort Atkinson, WI, USA) at developmental stages 44-47, and raised to stages 48-49 on a 12/12 h light/dark cycle at 18 $^{\circ}$C. 

In the beginning of each experiment, a tadpole was put in a Petri dish (diameter of 10 cm) filled with 1-1.2 cm of tadpole rearing medium, placed on top of a CRT monitor, with two speakers connected to the Petri dish with short wooden struts \citep{james2015,truszkowski2017}, and kept there for 4 hours. The tadpole was visually isolated from the rest of the room with a cardboard box surrounding the apparatus. For Control and Sound groups, the monitor was on, but showed a uniform 50\% gray background. For Flash, Sync and Async groups the screen showed a black-and-white checkerboard pattern, with each square in the pattern being 14 mm wide; this pattern flipped (inverted) every 1 second. For the Looming group, the inversion of the pattern was not instantaneous, but lasted for one second, with old black squares linearly shrinking into white background, and new black squares appearing and linearly expanding in the middle of each white square (Figure 1C). The stimulation program was written in JavaScript, using the p5.js library \citep{mccarthy2015}, and is available at \url{http://faculty.bard.edu/~akhakhal/checker_flash_ding.html} . For Sound, Sync, and Async groups a broad-spectrum sound click was delivered through the speakers every 1 second, with left and right speakers playing the same waveform, but inverted. Formally the click was generated as a 5 ms pulse of 100 Hz sine wave, but it was also distorted by the non-linearities in the system. The sound volume was calibrated to be about 2 times higher than the threshold volume, which means that it reliably evoked startle responses with about 80\% success ratio, at least at the beginning for the conditioning protocol. For the Sync group, the sound clicks and the checkerboard inversions were synchronized, while for the Async group they were offset by 500 ms (half a period).

\subsection*{Electrophysiology}

Immediately after sensory conditioning, tadpoles were anesthetized in 0.02\% tricaine methanesulfanate (MS-222). Dorsal commissures were cut, the brain was dissected out \citep{aizenman2003,ciarleglio2015}, and placed in the recording chamber filled with artificial cerebrospinal fluid (in mM: 115 NaCl, 4 KCl, 3 CaCl2, 3 MgCl2, 5 HEPES, 10 glucose, 10 $\mu$M glycine; pH 7.2, osmolarity 255 mOsm). All chemicals were obtained from Sigma (Sigma-Aldrich, St. Louis, MO). The ventricular membrane was removed (suctioned) using a broken glass electrode. Cells were visualized with a Nikon (Tokyo, Japan) Eclipse FN1 light microscope with a 40x water immersion objective. Recordings were restricted to the middle of the tectum, as in earlier studies \citep{ciarleglio2015}, from 25\% to 53\% of brain half-width medially from the lateral edge, and from 36\% to 69\% of tectum length rostrally from the caudal edge of the tectum (Figure 1A). Care was taken to record only from “deep” primary tectal cells (that are located superficially in our preparation), and not from MV cells \citep{pratt2009} or superficial layer cells (that are located deep in the tectum in our preparation) \citep{liu2016}. Glass electrodes (1.5x0.86 mm borosilicate glass; Sutter instruments, Novato, CA) were pulled on a Sutter P-1000 puller (Sutter instruments), to a tip resistance of 8-12 MOhm. The elecrodes were filled with intracellular saline (in mM: 100 K-gluconate, 5 NaCl, 8 KCl, 1.5 MgCl2, 20 HEPES, 10 EGTA, 2 ATP, 0.3 GTP; pH 7.2, osmolarity 255 mOsm). Electrodes were placed in an Axon headstage (Molecular Devices, Sunnyvale, CA), controlled by a motorized micromanipulator (MX7600, Siskiyou, Grants Pass, OR). Whole cell patch clamp was established as usual \citep{ciarleglio2015}, with typical final access resistance of 30 MOhm, and membrane resistance Rm of 0.33 GOhm. Signals were measured with an Axon Instruments MultiClamp 700B amplifier (Axon Instruments, Foster City, CA), filtered with a 5 kHz band-pass filter, and digitized at 10 kHz a CED Power1401-3 Digitizer (Cambridge Electronic Design; Cambridge, England). For synaptic stimulation, a bipolar stimulating electrode (Warner Instruments, Hamden, CT) was placed on the optic chiasm [Wu 1996]; stimuli were controlled by the CED digitizer, and were delivered by A.M.P.I. stimulus isolator (AMPI, Jerusalem, Israel).

Each neuron was subjected to a series of electrophysiological measurements protocols (see below for details), closely matching experimental protocols from \citep{ciarleglio2015}. For each cell, we measured membrane resistance Rm and capacitance Cm in voltage clamp mode, and then (1) ran a series of voltage steps to measure ionic currents; (2) in current clamp mode, ran a series of current steps to assess cell spiking; (3) in dynamic clamp mode, subjected the cell to different conductance injections; (4) finally, if the cell was still in a good health, ran a synaptic protocol with optic chiasm stimulation. All data was processed offline using custom Matlab scripts (Mathworks, Natick, MA), and analyzed in R. In total, we recorded from 188 neurons in 35 tadpoles; of these, 159 cells had readings from all 4 protocols, while 12 lacked synaptic recordings; these 12 cells were scattered across all 6 experimental groups. After the recording was over, the position of each recorded cell was visualized with a 10x microscope, marked on a screen, measured in medial and rostral directions, relative to the most lateral caudal point of the tectum, and converted into percentage \citep{hamodi2014}.

\subsection*{Voltage steps protocol}

The baseline membrane potential was set at $-$60 mV (in this manuscript, the voltages are not adjusted for junction potential, which is expected to be equal to $-$12 mV for this combination of external and internal solutions). After Cm and Rm were measured with a standard seal test, cells were subjected to 11 voltage steps (square pulses), each 500 ms long, and 10 mV higher than the previous one, with 500 ms of baseline voltage between the steps. Each trial also contained a 50 ms long test pre-step of $-$10 mV relative to the baseline. During analysis, we averaged transition currents evoked by the leading and trailing edges of the pre-step, then scaled, and subtracted them from the current responses to the main step. For remaining active currents, we measured average currents during 0.4-2.7 ms after the step (Na current), 5.7-19.7 ms after the step (Kt, or transient potassium current), and 430-490 ms (Ks, or stable potassium current). This approach is standard for recordings from the Xenpus tectum, as ionic currents are slow enough to be separated temporally \citep{aizenman2003}. The ionic conductances were quantified as is \citep{ciarleglio2015}. For each cell, the values of current as a function of voltage were fit with an empirical parametric equation:

$$I(v) = c \cdot \exp(x/b_1)/(1+\exp(-(a-x)/b_2))$$ 

\noindent for Na and Kt currents (sigmoid, followed by exponential decay, inactivating), and a different equation: 

$$I(v)=\max(0,\exp((x-a)/b)-e)\cdot c+d$$ 

\noindent for Ks current (a shifted piece of exponentially increasing curve with its lower part cut off; not inactivating). For equations with inactivation, we used its $I_{max}$ value at a measure of amplitude, and $v_a$ on the rising front such $I(v_a)=I_{max}/2$ as the threshold potential. For curves without inactivation (Ks) we used $I_{max}$ and the first non-zero point ($v_a=a+\log(e)\cdot b$) for the same purpose.

As a preliminary verification of our results, we compared the overall structure of our new dataset with the dataset from the 2015 study \citep{ciarleglio2015}. The eight cellular parameters described above showed similar pattern of coordination in both datasets: 23 pairwise correlations out of 35 total were significant (p$<$0.05) in this dataset, compared to 21 out of 35 in the 2015 study. The average absolute value of correlation coefficient was r=0.38 in this study, compared to r=0.32 in 2015 study. This suggests that the datasets are similar and representative of true internal variability in the tectum.

\subsection*{Current steps protocol}

For the current steps protocol, we switched each cell to current clamp mode, and adjusted the stable holding current to bring the resting membrane potential to about $-$60 mV. We then subjected the cell to 10 current pulses 150 ms each, delivered every 1 s, with currents ranging from 0 to 180 pA, in 20 pA increments. Cells that did not produce at least one spike in this experiment were considered not-excitable, and were not included in the dataset. The largest number of spikes produced in response to a single current injection was estimated offline, manually, using a custom Matlab data browser that blinded the researcher to the identity of the cell. As a control, spikes were also detected automatically, using the filtering and thresholding approach that was used in \citep{ciarleglio2015}; in 78\% of cells both manual and automated estimations matched, in remaining 22\% of cells the mismatch was either due to artifacts on the rising front being auto-detected as spikes, or due to spike broadening that fell below the threshold for the adaptive filter. The number of cells in which manual spike detection disagreed with automated detection did not differ across groups (6.1$\pm$1.7; p=0.5, exact Fisher test).

\subsection*{Dynamic clamp protocol}

For dynamic clamp experiments, each cell was held at $-$50 mV baseline potential, and was stimulated with 12 different “conductance injections”, each repeated 5 times. Conductance curves were generated with a formula $G=g\frac{t}{\tau}\exp(1-\frac{t}{\tau})$ , known as “alpha synapse”, where $g$ and $\tau$ are conductance and decay parameters respectively \citep{destexhe1994}. We used four different values of $\tau$, to represent four typical patterns of synaptic activation: 20 ms, corresponding to the total curve length (decay to 10\ of the peak value) of about 100 ms, to approximate short monosynaptic inputs \citep{ciarleglio2015}; 40 ms, corresponding to the total curve length of about 200 ms, as for a typical in-vitro stimulus with polysynaptic activation \citep{xu2011}; 100 ms, to mimic in-vivo inputs to the tectum in response to abrupt disappearance of light (“dark-flash”) \citep{khakhalin2014}; and 200 ms, to mimic retinal inputs in response to a 1 second-long linear looming stimulus \citep{khakhalin2014}. Actual decay times to 10\% of peak amplitude were 98, 196, 489, and 978 ms respectively. The value of $g$ was adjusted so that conductance curves peaked at 3 target conductances of 0.2, 0.5, and 1 nS. With the cell clamped at $-$50 mV, these conductances would have induced currents that peaked at 10, 25, and 50 pA respectively, matching the range of peak synaptic currents observed in \citep{xu2011,khakhalin2014,ciarleglio2015}. 

For each cell, for each of 60 trials, spikes were counted manually, blindly, and independently by both authors, using a custom Matlab data browser script. There was a 98.5\% agreement between spike number estimations on a trace-by-trace basis. All cases of disagreement (usually $\pm$1 spike) were due to later action potentials becoming broader and smaller in amplitude, which made them ambiguous. We ran sensitivity analyses of main effects reported in this paper separately on both estimations, and got qualitatively identical results. Numerically, we went with consensus numbers that in each case followed the higher estimation for the number of spikes.

To quantify the “shape” of spiking responses to conductances of different duration (temporal tuning), for every cell we encoded curve duration as an ordinal value (from 1 to 4), fit the spike data as a function of response duration with a quadratic formula ($y = ax^2 + bx +c$), and used the quadratic coefficient $a$ as the measure of response non-linearity. While the units and absolute values of this coefficient are not interpretable, it captures the shape of the response curve well, and allows for easy comparisons between cells (Figure 2A). The case of $a=0$ corresponds to spiking output linearly increasing with duration increase; $a>0$ means supralinear preference for long conductances (curving up); about $-0.25<a<0$ corresponds to a plateau-shaped curves, while $a<-0.25$ would mean heavy spike inactivation for longer conductance injections. 

\subsection*{Synaptic recordings}

For synaptic recordings, we switched cells back to voltage clamp mode, holding the membrane potential at $-$45 mV, to isolate excitatory synaptic currents. Optic chiasm shocks were delivered 10 times, every 20 s, with stimulation strength between 0.05 and 0.4 mA, and pulse length of 0.2 us. In each experiment, we would first find stimulation strength that evoked consistent synaptic responses in the cell we patched, then increased it by 20\%, and kept it constant for all cells recorded from this brain. Recordings were processed offline; for each trial we used the average current between 5 and 15 ms as a measure of monosynaptic response amplitude, and current between 15 and 145 ms as a measure of polysynaptic response amplitude \citep{ciarleglio2015}. The weighted duration of synaptic responses was calculated as the “center of mass” under the first 700 ms of the curve:

$$\displaystyle l=\int\limits_{0 \leq t \leq T}{I(t)\, t \, dt} \; \Big/ \int\limits_{0 \leq t \leq T}{I(t) \, dt}$$

\subsection*{Statistics and reporting}

To analyze the numbers of spikes observed in dynamic clamp experiments (Figure 1), we first averaged the number of generated spikes across 5 protocol repetitions, for each cell, and for each combination of conductance curve duration and amplitude, which resulted in 12 values per cell. Then we used sequential sum of squares analysis of variance with repeated measures. Both different conductance curve amplitudes and durations were represented as ordinal values (1 to 3 for amplitudes, 1 to 4 for durations). Differences between experimental groups were assessed as interactions between these ordinal values and the factor variable encoding the experimental group, as we were interested in response shapes (reflected by interactions) rather than average values of spikiness (reflected by independent terms). Cell ids were included in the analysis as a fixed factor for repeated measures analysis of variance (also equivalent to analysis of variance with blocking). To verify the validity of this approach, we also ran a maximal likelihood mixed-effects model with type III interaction terms, as implemented in R package “lmer”, with “lmerTest” extension to get access to Satterthwaite degrees of freedom and p-value estimations \citep{kuznetsova2017}. The results of both methods were numerically very similar.

For the comparison or summative descriptions of tuning, and other electrophysiological cell parameters between experimental groups, we report p-values of fixed effect sequential sum of squares linear model (ancova), in which rostral and medial coordinates of each cell within the tectum are included as covariates, and experimental group is used as the main factor. All comparisons and correlations between cell parameters are performed on values corrected for cell position within the tectum. Position correction was based on a two-way linear regression model without interaction. For five variables that were distributed extremely non-normally, this correction for position was performed on transformed values (original values were transformed to normally distributed proxy values, linearly adjusted, and then transformed back): for early and late mean synaptic amplitudes we used a transformation $a'=log(1-a)$; for the variability of synaptic amplitudes $s'=log(1+s)$; for temporal tuning $y_t' = \sqrt{y_t}$ . Where appropriate, we performed the analysis with and without extreme outliers, and reported the difference. All analyses presented in the paper were also verified in mixed model analyses, with animal id included as a random factor; the results of these mixed models analyses were similar to that of a fixed model, and are not reported.

\nolinenumbers
\bibliographystyle{apalike} % For author-year
%\bibliographystyle{unsrtnat} % For Nature-style
\bibliography{refs}

%TC:endignore
\end{document}
